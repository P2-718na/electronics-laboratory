\section{Introduzione}\label{sec:textscopo}
Un full adder è un dispositivo che prende come input due bit da sommare più un bit di carry e restituisce in output
un bit di somma e un bit di carry.
L'equazione booleana di questo dispositivo è riportata in appendice \ref{sec:1bfa-equazione}.
Questo dispositivo si può realizzare usando due porte \textsc{exor}, due porte \textsc{and} e una porta \textsc{or}, così come
riportato in figura \ref{fig:circuito}.
È possibile realizzare una porta \textsc{and} e una porta \textsc{exor} in logica positiva usando due diodi e una
resistenza.

