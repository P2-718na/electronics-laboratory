\section{Risultati}\label{sec:risultati}
Le misure sono state svolte sotto le condizioni riportate in tabella \ref{tab:livelli-logici}.
I risultati dei test delle porte logiche sono riportati di seguente, nelle tabelle \ref{tab:and-fulladder}, \ref{tab:exor-fulladder} e \ref{tab:or-fulladder}.
I risultati delle misure dello 1-bit fulla dder sono riportati in tabella \ref{tab:fulladder}.
I risultati delle misure delle porte \textsc{and} e \textsc{or} sono riportati rispettivamente in tabella \ref{tab:and-logicapositiva} e \ref{tab:or-logicapositiva}.

\begin{table}[H]
  \centering
  \begin{subtable}[H]{0.5\textwidth}
    \centering
    \begin{tabular}[t]{c  c | c  c | c  c}
      \hline
      A & B & V(A) & V(B) & Out & V(Out) (V)\\
      \hline
      0 & 0 & $V_{0}$ & $V_{0}$ & 0 & $0.068 \pm 0.010$ \\
      0 & 1 & $V_{0}$ & $V_{1}$ & 0 & $0.068 \pm 0.010$ \\
      1 & 0 & $V_{1}$ & $V_{0}$ & 0 & $0.068 \pm 0.010$ \\
      1 & 1 & $V_{1}$ & $V_{1}$ & 1 & $4.10 \pm 0.02$ \\
      \hline
    \end{tabular}
  \end{subtable}

  \vspace{.5cm}

  \begin{subtable}[H]{0.5\textwidth}
    \centering
    \begin{tabular}[t]{c  c | c  c | c  c}
      \hline
      A & B & V(A) & V(B) & Out & V(Out) (V)\\
      \hline
      0 & 0 & $V_{0}$ & $V_{0}$ & 0 & $0.070 \pm 0.012$ \\
      0 & 1 & $V_{0}$ & $V_{1}$ & 0 & $0.071 \pm 0.010$ \\
      1 & 0 & $V_{1}$ & $V_{0}$ & 0 & $0.068 \pm 0.010$ \\
      1 & 1 & $V_{1}$ & $V_{1}$ & 1 & $4.09 \pm 0.02$ \\
      \hline
    \end{tabular}
  \end{subtable}
  \caption{\emph{Tavole di verità delle porte \textsc{and} corrispondenti ai pin 1-2-3 (sopra) e 4-5-6 (sotto) dell'integrato 7408. I valori di $V_{0}$ e $V_{1}$ sono quelli riportati in tabella \ref{tab:livelli-logici}} sotto la dicitura "F.A.".}
  \label{tab:and-fulladder}
\end{table}

\begin{table}[H]
  \centering
  \begin{subtable}[H]{0.5\textwidth}
    \centering
    \begin{tabular}[t]{c  c | c  c | c  c}
      \hline
      A & B & V(A) & V(B) & Out & V(Out) (V)\\
      \hline
      0 & 0 & $V_{0}$ & $V_{0}$ & 0 & $0.364 \pm 0.011$ \\
      0 & 1 & $V_{0}$ & $V_{1}$ & 1 & $5.03 \pm 0.03$ \\
      1 & 0 & $V_{1}$ & $V_{0}$ & 1 & $5.03 \pm 0.03$ \\
      1 & 1 & $V_{1}$ & $V_{1}$ & 0 & $0.400 \pm 0.011$ \\
      \hline
    \end{tabular}
  \end{subtable}

  \vspace{.5cm}

  \begin{subtable}[H]{0.5\textwidth}
    \centering
    \begin{tabular}[t]{c  c | c  c | c  c}
      \hline
      A & B & V(A) & V(B) & Out & V(Out) (V)\\
      \hline
      0 & 0 & $V_{0}$ & $V_{0}$ & 0 & $0.402 \pm 0.011$ \\
      0 & 1 & $V_{0}$ & $V_{1}$ & 1 & $5.03 \pm 0.03$ \\
      1 & 0 & $V_{1}$ & $V_{0}$ & 1 & $5.03 \pm 0.03$ \\
      1 & 1 & $V_{1}$ & $V_{1}$ & 0 & $0.443 \pm 0.011$ \\
      \hline
    \end{tabular}
  \end{subtable}
  \caption{\emph{Tavole di verità delle porte \textsc{exor} corrispondenti ai pin 13-12-11 (sopra) e 10-9-8 (sotto) dell'integrato 7408. I valori di $V_{0}$ e $V_{1}$ sono quelli riportati in tabella \ref{tab:livelli-logici}} sotto la dicitura "F.A.".}
  \label{tab:exor-fulladder}
\end{table}

\begin{table}[H]
  \centering
  \begin{tabular}[t]{c  c | c  c | c  c}
    \hline
    A & B & V(A) & V(B) & Out & V(Out) (V)\\
    \hline
    0 & 0 & $V_{0}$ & $V_{0}$ & 0 & $0.081 \pm 0.010$ \\
    0 & 1 & $V_{0}$ & $V_{1}$ & 1 & $4.14 \pm 0.02$ \\
    1 & 0 & $V_{1}$ & $V_{0}$ & 1 & $4.14 \pm 0.02$ \\
    1 & 1 & $V_{1}$ & $V_{1}$ & 1 & $4.15 \pm 0.02$ \\
    \hline
  \end{tabular}
  \caption{\emph{Tavole di verità della porta \textsc{or} corrispondenti ai pin 1-2-3 dell'integrato 7408. I valori di $V_{0}$ e $V_{1}$ sono quelli riportati in tabella \ref{tab:livelli-logici}} sotto la dicitura "F.A.".}
  \label{tab:or-fulladder}
\end{table}

\begin{table}[H]
  \centering
  \begin{tabular}[t]{c  c  c | c  c  c | c  c | c  c}
    \hline
    A & B & $C_{in}$ & V(A) & V(B) & V($C_{in}$) & S & V(S) (V) & $C_{out}$ & V($C_{out}$) (V) \\
    \hline
    0 & 0 & 0 & $V_{0}$ & $V_{0}$ & $V_{0}$ & 0 & $0.372 \pm 0.011$   & 0 & $0.084 \pm 0.010 $\\
    0 & 0 & 1 & $V_{0}$ & $V_{0}$ & $V_{1}$ & 1 & $5.03 \pm 0.03$   & 0 & $0.086 \pm 0.010$\\
    0 & 1 & 0 & $V_{0}$ & $V_{1}$ & $V_{0}$ & 1 & $5.04 \pm 0.03$   & 0 & $0.082 \pm 0.010$\\
    0 & 1 & 1 & $V_{0}$ & $V_{1}$ & $V_{1}$ & 0 & $0.403 \pm 0.011$ & 1 & $4.15 \pm 0.02$\\
    1 & 0 & 0 & $V_{1}$ & $V_{0}$ & $V_{0}$ & 1 & $5.03 \pm 0.03$   & 0 & $0.083 \pm 0.010$\\
    1 & 0 & 1 & $V_{1}$ & $V_{0}$ & $V_{1}$ & 0 & $0.403 \pm 0.011$ & 1 & $4.15 \pm 0.02$\\
    1 & 1 & 0 & $V_{1}$ & $V_{1}$ & $V_{0}$ & 0 & $0.374 \pm 0.011$ & 1 & $4.14 \pm 0.02$\\
    1 & 1 & 1 & $V_{1}$ & $V_{1}$ & $V_{1}$ & 1 & $5.03 \pm 0.03$   & 1 & $4.14 \pm 0.02$\\
    \hline
  \end{tabular}
  \caption{\emph{Tavole di verità del circuito \emph{1-bit full adder}. I valori di $V_{0}$ e $V_{1}$ sono quelli riportati in tabella \ref{tab:livelli-logici}} sotto la dicitura "F.A.".}
  \label{tab:fulladder}
\end{table}

\begin{table}[H]
  \centering
  \begin{tabular}[t]{c  c | c  c | c  c}
    \hline
    $V_{1}$ & $V_{2}$ & $V_{1}$ & $V_{2}$ & $V_{out}$ & $V_{out}$ (V) \\
    \hline
    0 & 0 & $V_{0}$ & $V_{0}$ & 0 & $0.441 \pm 0.011$ \\
    0 & 1 & $V_{0}$ & $V_{1}$ & 0 & $0.461 \pm 0.011$ \\
    1 & 0 & $V_{1}$ & $V_{0}$ & 0 & $0.518 \pm 0.011$ \\
    1 & 1 & $V_{1}$ & $V_{1}$ & 1 & $4.97 \pm 0.02$ \\
    \hline
  \end{tabular}
  \caption{\emph{Tavole di verità della porta \emph{AND} realizzata con un circuito in logica positiva. I valori di $V_{0}$ e $V_{1}$ sono quelli riportati in tabella \ref{tab:livelli-logici}} sotto la dicitura "logica positiva".}
  \label{tab:and-logicapositiva}
\end{table}

\begin{table}[H]
  \centering
  \begin{tabular}[t]{c  c | c  c | c  c}
    \hline
    $V_{1}$ & $V_{2}$ & $V_{1}$ & $V_{2}$ & $V_{out}$ & $V_{out}$ (V) \\
    \hline
    0 & 0 & $V_{0}$ & $V_{0}$ & 0 & $0.000 \pm 0.010$ \\
    0 & 1 & $V_{0}$ & $V_{1}$ & 1 & $4.52 \pm 0.02$ \\
    1 & 0 & $V_{1}$ & $V_{0}$ & 1 & $4.57 \pm 0.02$ \\
    1 & 1 & $V_{1}$ & $V_{1}$ & 1 & $4.59 \pm 0.02$ \\
    \hline
  \end{tabular}
  \caption{\emph{Tavole di verità della porta \emph{OR} realizzata con un circuito in logica positiva. I valori di $V_{0}$ e $V_{1}$ sono quelli riportati in tabella \ref{tab:livelli-logici}} sotto la dicitura "logica positiva".}
  \label{tab:or-logicapositiva}
\end{table}
