\section{Risultati}\label{sec:risultati}
  Valori di V0 e V1 %todo

 %todo
qui ci va la tavola di verità di test delle porte, con tanto di voltaggi in uscita (questo ci servirà dopo per circuito sommatore)

misura resistenzxa di pullup

\begin{table}[H]
  \centering
  \begin{subtable}[c]{0.5\textwidth}
    \centering
    \begin{tabular}[t]{c  c | c  c | c  c}
      \hline
      A & B & V(A) & V(B) & Out & V(Out) \\
      \hline
      0 & 0 & $V_{0}$ & $V_{0}$ & 0 & $(0.068 \pm 0.010) \: V$ \\
      0 & 1 & $V_{0}$ & $V_{1}$ & 0 & $(0.068 \pm 0.010) \: V$ \\
      1 & 0 & $V_{1}$ & $V_{0}$ & 0 & $(0.068 \pm 0.010) \: V$ \\
      1 & 1 & $V_{1}$ & $V_{1}$ & 1 & $(4.10 \pm 0.02) \: V$ \\
      \hline
    \end{tabular}
  \end{subtable}
  \begin{subtable}[c]{0.5\textwidth}
    \centering
    \begin{tabular}[t]{c  c | c  c | c  c}
      \hline
      A & B & V(A) & V(B) & Out & V(Out) \\
      \hline
      0 & 0 & $V_{0}$ & $V_{0}$ & 0 & $(0.070 \pm 0.012) \: V$ \\
      0 & 1 & $V_{0}$ & $V_{1}$ & 0 & $(0.071 \pm 0.010) \: V$ \\
      1 & 0 & $V_{1}$ & $V_{0}$ & 0 & $(0.068 \pm 0.010) \: V$ \\
      1 & 1 & $V_{1}$ & $V_{1}$ & 1 & $(4.09 \pm 0.02) \: V$ \\
      \hline
    \end{tabular}
  \end{subtable}
  \caption{\emph{Tavole di verità delle porte \emph{AND} corrispondenti ai pin 1-2-3 (sinistra) e 4-5-6 (destra) dell'integrato 7408. I valori di $V_{0}$ e $V_{1}$ sono quelli riportati in tabella \ref{tab:livelli-logici}} sotto la dicitura "F.A."}
  \label{tab:and-fulladder}
\end{table}

\begin{table}[H]
  \centering
  \begin{subtable}[c]{0.5\textwidth}
    \centering
    \begin{tabular}[t]{c  c | c  c | c  c}
      \hline
      A & B & V(A) & V(B) & Out & V(Out) \\
      \hline
      0 & 0 & $V_{0}$ & $V_{0}$ & 0 & $(0.364 \pm 0.011) \: V$ \\
      0 & 1 & $V_{0}$ & $V_{1}$ & 1 & $(5.03 \pm 0.03) \: V$ \\
      1 & 0 & $V_{1}$ & $V_{0}$ & 1 & $(5.03 \pm 0.03) \: V$ \\
      1 & 1 & $V_{1}$ & $V_{1}$ & o & $(0.400 \pm 0.011) \: V$ \\
      \hline
    \end{tabular}
  \end{subtable}
  \begin{subtable}[c]{0.5\textwidth}
    \centering
    \begin{tabular}[t]{c  c | c  c | c  c}
      \hline
      A & B & V(A) & V(B) & Out & V(Out) \\
      \hline
      0 & 0 & $V_{0}$ & $V_{0}$ & 0 & $(0.402 \pm 0.011) \: V$ \\
      0 & 1 & $V_{0}$ & $V_{1}$ & 1 & $(5.03 \pm 0.03) \: V$ \\
      1 & 0 & $V_{1}$ & $V_{0}$ & 1 & $(5.03 \pm 0.03) \: V$ \\
      1 & 1 & $V_{1}$ & $V_{1}$ & o & $(0.443 \pm 0.011) \: V$ \\
      \hline
    \end{tabular}
  \end{subtable}
  \caption{\emph{Tavole di verità delle porte \emph{EXOR} corrispondenti ai pin 13-12-11 (sinistra) e 10-9-8 (destra) dell'integrato 7408. I valori di $V_{0}$ e $V_{1}$ sono quelli riportati in tabella \ref{tab:livelli-logici}} sotto la dicitura "F.A."}
  \label{tab:exor-fulladder}
\end{table}

\begin{table}[H]
  \centering
  \begin{tabular}[t]{c  c | c  c | c  c}
    \hline
    A & B & V(A) & V(B) & Out & V(Out) \\
    \hline
    0 & 0 & $V_{0}$ & $V_{0}$ & 0 & $(0.081 \pm 0.010) \: V$ \\
    0 & 1 & $V_{0}$ & $V_{1}$ & 1 & $(4.14 \pm 0.02) \: V$ \\
    1 & 0 & $V_{1}$ & $V_{0}$ & 1 & $(4.14 \pm 0.02) \: V$ \\
    1 & 1 & $V_{1}$ & $V_{1}$ & 1 & $(4.15 \pm 0.02) \: V$ \\
    \hline
  \end{tabular}
  \caption{\emph{Tavole di verità della porta \emph{OR} corrispondenti ai pin 1-2-3 dell'integrato 7408. I valori di $V_{0}$ e $V_{1}$ sono quelli riportati in tabella \ref{tab:livelli-logici}} sotto la dicitura "F.A."}
  \label{tab:or-fulladder}
\end{table}

\begin{table}[H]
  \centering
  \begin{tabular}[t]{c  c  c | c  c  c | c  c | c  c}
    \hline
    A & B & $C_{in}$ & V(A) & V(B) & V($C_{in}$) & S & V(S) & $C_{out}$ & V($C_{out}$) \\
    \hline
    0 & 0 & 0 & $V_{0}$ & $V_{0}$ & $V_{0}$ & 0 & $(0.372 \pm 0.011) \: V$ & 0 & 0.084,0.010252
    0 & 0 & 1 & $V_{0}$ & $V_{0}$ & $V_{1}$ & 1 & $(5.03 \pm 0.03) \: V$ & 0 & 0.086,0.010258
    0 & 1 & 0 & $V_{0}$ & $V_{1}$ & $V_{0}$ & 1 & $(5.04 \pm 0.03) \: V$ & 0 & 0.082,0.010246
    0 & 1 & 1 & $V_{0}$ & $V_{1}$ & $V_{1}$ & 0 & $(0.403 \pm 0.011) \: V$ & 1 & 4.15,0.02245
    1 & 0 & 0 & $V_{1}$ & $V_{0}$ & $V_{0}$ & 1 & $(5.03 \pm 0.03) \: V$ & 0 & 0.083,0.010249
    1 & 0 & 1 & $V_{1}$ & $V_{0}$ & $V_{1}$ & 0 & $(0.403 \pm 0.011) \: V$ & 1 & 4.15,0.02245
    1 & 1 & 0 & $V_{1}$ & $V_{1}$ & $V_{0}$ & 0 & $(0.374 \pm 0.011) \: V$ & 1 & 4.14,0.02242
    1 & 1 & 1 & $V_{1}$ & $V_{1}$ & $V_{1}$ & 1 & $(5.03 \pm 0.03) \: V$ & 1 & 4.14,0.02242
    \hline
  \end{tabular}
  \caption{\emph{Tavole di verità del circuito \emph{1-bit full adder}. I valori di $V_{0}$ e $V_{1}$ sono quelli riportati in tabella \ref{tab:livelli-logici}} sotto la dicitura "F.A."}
  \label{tab:fulladder}
\end{table}

\begin{table}[H]
  \centering
  \begin{tabular}[t]{c  c | c  c | c  c}
    \hline
    $V_{1}$ & $V_{2}$ & $V_{1}$ & $V_{2}$ & $V_{out}$ & $V_{out}$ (V) \\
    \hline
    0 & 0 & $V_{0}$ & $V_{0}$ & 0 & $0.441 \pm 0.011$ \\
    0 & 1 & $V_{0}$ & $V_{1}$ & 0 & $0.461 \pm 0.011$ \\
    1 & 0 & $V_{1}$ & $V_{0}$ & 0 & $0.518 \pm 0.011$ \\
    1 & 1 & $V_{1}$ & $V_{1}$ & 1 & $4.97 \pm 0.02$ \\
    \hline
  \end{tabular}
  \caption{\emph{Tavole di verità della porta \emph{AND} realizzata con un circuito in logica positiva. I valori di $V_{0}$ e $V_{1}$ sono quelli riportati in tabella \ref{tab:livelli-logici}} sotto la dicitura "logica positiva"}
  \label{tab:and-logicapositiva}
\end{table}

\begin{table}[H]
  \centering
  \begin{tabular}[t]{c  c | c  c | c  c}
    \hline
    $V_{1}$ & $V_{2}$ & $V_{1}$ & $V_{2}$ & $V_{out}$ & $V_{out}$ (V) \\
    \hline
    0 & 0 & $V_{0}$ & $V_{0}$ & 0 & $0.000 \pm 0.010$ \\
    0 & 1 & $V_{0}$ & $V_{1}$ & 1 & $4.52 \pm 0.02$ \\
    1 & 0 & $V_{1}$ & $V_{0}$ & 1 & $4.57 \pm 0.02$ \\
    1 & 1 & $V_{1}$ & $V_{1}$ & 1 & $4.59 \pm 0.02$ \\
    \hline
  \end{tabular}
  \caption{\emph{Tavole di verità della porta \emph{OR} realizzata con un circuito in logica positiva. I valori di $V_{0}$ e $V_{1}$ sono quelli riportati in tabella \ref{tab:livelli-logici}} sotto la dicitura "logica positiva"}
  \label{tab:or-logicapositiva}
\end{table}