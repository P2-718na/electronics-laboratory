\section{Apparato sperimentale}\label{sec:apparato-sperimentale}
\subsection{Schema del circuito}\label{subsec:schema-circuito}

GLI XOR SONO OPEN COLLECTOR => SUll'usciTA DI OGNI XOR BISOGNA AGGIUNGERE CIRCUITO DI PULLUP

%figura\begin{figure}[h]
%figura  \centering
%figura  \includegraphics[width=10cm]{../assets/circuito.drawio.pdf}
%figura  \caption{
%figura    \emph{
%figura      Schema del circuito tosatore.
%figura    }
%figura  }
%figura  \label{fig:circuito}
%figura\end{figure} %todo

Il circuito che abbiamo realizzato è schematizzato in figura \ref{fig:circuito}.
È strutturato come segue:
\begin{enumerate}
  \item%
  sprema %todo
\end{enumerate}
%todo

\subsection{Materiale e strumenti usati}\label{subsec:materiali}
Segue una lista del materiale e degli strumenti usati durante la prova:
\begin{itemize}
  \item%
  Oscilloscopio analogico, modello: \emph{GW Instek GOS-652}.
  \item%
  Multimetro digitale, modello: \emph{ISO-TECH IDM 105}.
  \item%
  Generatore di tensione, modello: \emph{Aim-TTi EB2025T}.
  \item%
  Generatore di onda sinusoidale, modello: \emph{GFG-8017G}.
  \item%
  Sonda per oscilloscopio.
  \item%
  Connettori vari (connettori a banana, cavi per la scheda millefori).
  \item%
  Circuiti integrati per \sc{or}, \sc{and}, \sc{xor} (%todo modello).

\end{itemize}
