\documentclass[11pt, a4paper, twoside]{article}
\raggedbottom
% questi due pacchetti servono per indicare la codifica della tastiera usata e i caratteri che vogliamo usare.
% a noi non servono molto, ma metti che compiliamo con un computer russo?
\usepackage[T1]{fontenc}
\usepackage[utf8]{inputenc}
%non capisco il perché della metà dei pacchetti kek
\usepackage{blindtext}
\usepackage{geometry}
\usepackage{setspace}
\usepackage{titlesec}
\usepackage{indentfirst}
\usepackage{graphicx}
\usepackage[italian]{babel}
\usepackage{catchfile} % used in \getenv command
\usepackage{multicol}
\usepackage{amsmath}
\usepackage{subcaption}
\usepackage[hang, flushmargin, multiple, bottom]{footmisc}
\usepackage{float}
\usepackage{array}
\usepackage{booktabs}
\usepackage{url}
\usepackage{csvsimple}

\titlespacing*{\section}{0px}{3mm}{1mm}         %what purpouse?
\titlespacing*{\subsection}{0px}{3mm}{1mm}      %what purpouse?
\geometry{
  left=2cm,
  right=2cm,
  top=2cm,
  bottom=2cm
}
\setlength{\parindent}{10mm}
\graphicspath{ {./assets}, {../../assets} }

% Allow use of command \getenv{VARNAME}.
% Taken from: https://tex.stackexchange.com/questions/62010/can-i-access-system-environment-variables-from-latex-for-instance-home
\newcommand{\getenv}[2][]{
  \CatchFileEdef{\temp}{"|kpsewhich --var-value #2"}{\endlinechar=-1}%
  \if\relax\detokenize{#1}\relax\temp\else\let#1\temp\fi}

% Roman numerals
\newcommand{\rom}[1]{\uppercase\expandafter{\romannumeral #1\relax}}

% authors, date and title---------------------------------------------------------------
\author{Giuseppe Sguera \\ \getenv{MAT1} \and Matteo Bonacini \\ \getenv{MAT2}}
\date{\today}
\title{Misura della caratteristica di uscita di un BJT P-N-P on configurazione a emettitore comune}
%---------------------------------------------------------------------------------------
\begin{document}

    %\twocolumn[
    %  \begin{@twocolumnfalse}
        %\begin{center}
{
	{\Large
		{\textsc{Alma Mater Studiorum $\cdot$ Università di Bologna}}
	}
}
\rule[0.1cm]{18cm}{0.1mm}
\rule[0.5cm]{18cm}{0.6mm}
{\small
	{\bf SCUOLA DI SCIENZE\\
		Corso di Laurea in Fisica
	}
}
{\let\newpage\relax\maketitle}
\end{center}


    \maketitle

    \begin{abstract}\label{sec:abstract}
Abstract
\end{abstract}
 %todo
    % \end{@twocolumnfalse}
    %]

    \section{Introduzione}\label{sec:scopo}
La caratteristica di uscita di un transistor \emph{BJT} è descritta dalle
equazioni di Ebers-Moll. Noi siamo interessati al funzionamento del transistor in
regione attiva, per cui vale la seguente relazione:
\begin{equation}
  I_C = \alpha_F I_{ES} (e^{\frac{V_{BE}}{\eta V_T}} - 1)
  \label{eq:caratteristica}
\end{equation}
$I_C$ è la corrente uscente dal collettore, mentre $V_{BE}$ è la tensione tra base
ed emettitore. Si noti che, nella configurazione di questo esperimento, i valori di $I_C$
e $V_{BE}$ saranno negativi.
L'equazione \eqref{eq:caratteristica} prevede una rapida crescita esponenziale,
seguita da un tratto pressoché costante. Quello che si osserva nella realtà, invece, è che
il tratto presenta una leggera pendenza, dovuta a una dipendenza lineare da $V_{BE}$.
Questo fenomeno prende il nome di \emph{effetto Early}, ed è descritto dalla legge fenomenologica \eqref{eq:effetto-early}.
\begin{equation}
  I_C = \beta_F I_B \left( 1+ \frac {V_{CE}} {V_A} \right)
  \label{eq:effetto-early}
\end{equation}
In questo esperimento analizzeremo questo fenomeno, \emph{plottando} la caratteristica $I_V$
di un transistor e svolgendo un \emph{fit} lineare per misurarne tensione di Early,
resistenza di uscita, conduttanza di uscita e guadagno di corrente. Il \emph{fit} è svolto
con la seguente equazione:
\begin{equation}
  V_{CE} = a + b I_C
  \label{eq:fit}
\end{equation}
dove $a$ è la tensione di Early, $b$ è la resistenza di uscita e $\frac 1 b = g$ è la
conduttanza di uscita. Il guadagno di corrente $\beta$ è ottenuto fissando un valore
di $V_{CE}$ e calcolando il rapporto $\frac {\Delta I_C} {\Delta I_B}$ tra due differenti \emph{set}
di dati.
 %todo
    \section{Apparato sperimentale}\label{sec:apparato-sperimentale}
\subsection{Schema del circuito}\label{subsec:schema-circuito}

\begin{figure}[h]
  \centering
      \includegraphics[width=10cm]{../assets/circuito.drawio.pdf}
      \caption{
        \emph{
          Schema del circuito tosatore.
        }
      }
    \label{fig:circuito}
\end{figure} %todo

Il circuito che abbiamo realizzato è schematizzato in figura \ref{fig:circuito}.
È strutturato come segue:
\begin{enumerate}
  \item%
  Due potenziometri da $1k\Omega$ sono collegati a terra e a $\pm 5V$, rispettivamente.
  \item%
  Due diodi al silicio (germanio) sono collegati al pin centrale dei potenziometri.
  \item%
  Un generatore di onda sinusoidale è collegato a una resistenza variabile da $50k\Omega$.
  \item%
  Un canale dell'oscilloscopio vede l'output \emph{pulito} del generatore di funzione.
  \item%
  Un canale dell'oscilloscopio è collegato in parallelo ai due diodi e all'altro capo della resistenza variabile da $50k\Omega$.
\end{enumerate}
Il circuito così realizzato permette di regolare l'ampiezza dell'onda tosata e la curvatura
della porzione tosata.
Per fare ciò, bisogna agire rispettivamente sui due potenziometri da 1k$\Omega$
e sul potenziometro da 50k$\Omega$.

\subsection{Materiale e strumenti usati}\label{subsec:materiali}
Segue una lista del materiale e degli strumenti usati durante la prova:
\begin{itemize}
  \item%
  Oscilloscopio analogico, modello: \emph{GW Instek GOS-652}.
  \item%
  Multimetro digitale, modello: \emph{ISO-TECH IDM 105}.
  \item%
  Generatore di tensione, modello: \emph{Aim-TTi EB2025T}.
  \item%
  Generatore di onda sinusoidale, modello: \emph{GFG-8017G}.
  \item%
  Sonda per oscilloscopio.
  \item%
  Connettori vari (connettori a banana, cavi per la scheda millefori).
  \item%
  2 diodi al silicio
  \item%
  2 diodi al germanio
  \item
  2 potenziometri da $1k\Omega$.
  \item
  Potenziometro da $50k\Omega$.
\end{itemize}

    \section{Risultati}\label{sec:risultati}
  Valori di V0 e V1 %todo

 %todo
qui ci va la tavola di verità di test delle porte, con tanto di voltaggi in uscita (questo ci servirà dopo per circuito sommatore)

misura resistenzxa di pullup

\begin{table}[H]
  \centering
  \begin{subtable}[c]{0.5\textwidth}
    \centering
    \begin{tabular}[t]{c  c | c  c | c  c}
      \hline
      A & B & V(A) & V(B) & Out & V(Out) \\
      \hline
      0 & 0 & $V_{0}$ & $V_{0}$ & 0 & $(0.068 \pm 0.010) \: V$ \\
      0 & 1 & $V_{0}$ & $V_{1}$ & 0 & $(0.068 \pm 0.010) \: V$ \\
      1 & 0 & $V_{1}$ & $V_{0}$ & 0 & $(0.068 \pm 0.010) \: V$ \\
      1 & 1 & $V_{1}$ & $V_{1}$ & 1 & $(4.10 \pm 0.02) \: V$ \\
      \hline
    \end{tabular}
  \end{subtable}
  \begin{subtable}[c]{0.5\textwidth}
    \centering
    \begin{tabular}[t]{c  c | c  c | c  c}
      \hline
      A & B & V(A) & V(B) & Out & V(Out) \\
      \hline
      0 & 0 & $V_{0}$ & $V_{0}$ & 0 & $(0.070 \pm 0.012) \: V$ \\
      0 & 1 & $V_{0}$ & $V_{1}$ & 0 & $(0.071 \pm 0.010) \: V$ \\
      1 & 0 & $V_{1}$ & $V_{0}$ & 0 & $(0.068 \pm 0.010) \: V$ \\
      1 & 1 & $V_{1}$ & $V_{1}$ & 1 & $(4.09 \pm 0.02) \: V$ \\
      \hline
    \end{tabular}
  \end{subtable}
  \caption{\emph{Tavole di verità delle porte \emph{AND} corrispondenti ai pin 1-2-3 (sinistra) e 4-5-6 (destra) dell'integrato 7408. I valori di $V_{0}$ e $V_{1}$ sono quelli riportati in tabella \ref{tab:livelli-logici}} sotto la dicitura "F.A."}
  \label{tab:and-fulladder}
\end{table}

\begin{table}[H]
  \centering
  \begin{subtable}[c]{0.5\textwidth}
    \centering
    \begin{tabular}[t]{c  c | c  c | c  c}
      \hline
      A & B & V(A) & V(B) & Out & V(Out) \\
      \hline
      0 & 0 & $V_{0}$ & $V_{0}$ & 0 & $(0.364 \pm 0.011) \: V$ \\
      0 & 1 & $V_{0}$ & $V_{1}$ & 1 & $(5.03 \pm 0.03) \: V$ \\
      1 & 0 & $V_{1}$ & $V_{0}$ & 1 & $(5.03 \pm 0.03) \: V$ \\
      1 & 1 & $V_{1}$ & $V_{1}$ & o & $(0.400 \pm 0.011) \: V$ \\
      \hline
    \end{tabular}
  \end{subtable}
  \begin{subtable}[c]{0.5\textwidth}
    \centering
    \begin{tabular}[t]{c  c | c  c | c  c}
      \hline
      A & B & V(A) & V(B) & Out & V(Out) \\
      \hline
      0 & 0 & $V_{0}$ & $V_{0}$ & 0 & $(0.402 \pm 0.011) \: V$ \\
      0 & 1 & $V_{0}$ & $V_{1}$ & 1 & $(5.03 \pm 0.03) \: V$ \\
      1 & 0 & $V_{1}$ & $V_{0}$ & 1 & $(5.03 \pm 0.03) \: V$ \\
      1 & 1 & $V_{1}$ & $V_{1}$ & o & $(0.443 \pm 0.011) \: V$ \\
      \hline
    \end{tabular}
  \end{subtable}
  \caption{\emph{Tavole di verità delle porte \emph{EXOR} corrispondenti ai pin 13-12-11 (sinistra) e 10-9-8 (destra) dell'integrato 7408. I valori di $V_{0}$ e $V_{1}$ sono quelli riportati in tabella \ref{tab:livelli-logici}} sotto la dicitura "F.A."}
  \label{tab:exor-fulladder}
\end{table}

\begin{table}[H]
  \centering
  \begin{tabular}[t]{c  c | c  c | c  c}
    \hline
    A & B & V(A) & V(B) & Out & V(Out) \\
    \hline
    0 & 0 & $V_{0}$ & $V_{0}$ & 0 & $(0.081 \pm 0.010) \: V$ \\
    0 & 1 & $V_{0}$ & $V_{1}$ & 1 & $(4.14 \pm 0.02) \: V$ \\
    1 & 0 & $V_{1}$ & $V_{0}$ & 1 & $(4.14 \pm 0.02) \: V$ \\
    1 & 1 & $V_{1}$ & $V_{1}$ & 1 & $(4.15 \pm 0.02) \: V$ \\
    \hline
  \end{tabular}
  \caption{\emph{Tavole di verità della porta \emph{OR} corrispondenti ai pin 1-2-3 dell'integrato 7408. I valori di $V_{0}$ e $V_{1}$ sono quelli riportati in tabella \ref{tab:livelli-logici}} sotto la dicitura "F.A."}
  \label{tab:or-fulladder}
\end{table}

\begin{table}[H]
  \centering
  \begin{tabular}[t]{c  c  c | c  c  c | c  c | c  c}
    \hline
    A & B & $C_{in}$ & V(A) & V(B) & V($C_{in}$) & S & V(S) & $C_{out}$ & V($C_{out}$) \\
    \hline
    0 & 0 & 0 & $V_{0}$ & $V_{0}$ & $V_{0}$ & 0 & $(0.372 \pm 0.011) \: V$ & 0 & 0.084,0.010252
    0 & 0 & 1 & $V_{0}$ & $V_{0}$ & $V_{1}$ & 1 & $(5.03 \pm 0.03) \: V$ & 0 & 0.086,0.010258
    0 & 1 & 0 & $V_{0}$ & $V_{1}$ & $V_{0}$ & 1 & $(5.04 \pm 0.03) \: V$ & 0 & 0.082,0.010246
    0 & 1 & 1 & $V_{0}$ & $V_{1}$ & $V_{1}$ & 0 & $(0.403 \pm 0.011) \: V$ & 1 & 4.15,0.02245
    1 & 0 & 0 & $V_{1}$ & $V_{0}$ & $V_{0}$ & 1 & $(5.03 \pm 0.03) \: V$ & 0 & 0.083,0.010249
    1 & 0 & 1 & $V_{1}$ & $V_{0}$ & $V_{1}$ & 0 & $(0.403 \pm 0.011) \: V$ & 1 & 4.15,0.02245
    1 & 1 & 0 & $V_{1}$ & $V_{1}$ & $V_{0}$ & 0 & $(0.374 \pm 0.011) \: V$ & 1 & 4.14,0.02242
    1 & 1 & 1 & $V_{1}$ & $V_{1}$ & $V_{1}$ & 1 & $(5.03 \pm 0.03) \: V$ & 1 & 4.14,0.02242
    \hline
  \end{tabular}
  \caption{\emph{Tavole di verità del circuito \emph{1-bit full adder}. I valori di $V_{0}$ e $V_{1}$ sono quelli riportati in tabella \ref{tab:livelli-logici}} sotto la dicitura "F.A."}
  \label{tab:fulladder}
\end{table} %todo
    \section{Conclusioni}\label{sec:conclusioni}
Il circuito realizzato si comporta esattamente come atteso.
Il multiplexer rispetta i valori di tensione di output richiesti: $<0.5V$ per un
segnale \emph{low} e $>2.7V$ per un segnale \emph{high}.
Questo comportamento vale sia per un segnale fisso che per un segnale che varia nel tempo. %todo
    \appendix
  \textbf{\huge{Appendice}}
    \section{Incertezze strumentali}\label{sec:incertezze-strumentali}
      \subsection{Incertezze per l'oscilloscopio}\label{subsec:incertezze-oscilloscopio}
      Nel nostro caso, l'oscilloscopio è lo strumento che introduce maggior incertezza sulle misure. Sullo schermo dello strumento,
      siamo riusciti ad apprezzare variazioni di $\frac 1 {10}$ del valore scelto come fondo scala. Per calcolare l'incertezza $\sigma$ associata
      ad una singola misura, abbiamo usato la formula \eqref{eq:incertezza-oscilloscopio}:
      \begin{equation}
        \sigma_V = \sqrt{
          \sigma_\text{casuale}^2 + \sigma_\text{sistematica}^2
        }
        \label{eq:incertezza-oscilloscopio}
      \end{equation}

      L'incertezza sistematica è fornita dal produttore dello strumento, e ha un valore di $\sigma_\text{sistematica}=3\%$.
      Per valutare l'incertezza casuale, invece, abbiamo usato la formula \eqref{eq:incertezza-oscilloscopio-casuale}:

      \begin{equation}
        \sigma_\text{casuale} = \sigma_0 + \sigma_i = 2\sigma_\text{risoluzione}
        \label{eq:incertezza-oscilloscopio-casuale}
      \end{equation}
      dove $\sigma_0$ è l'incertezza dovuta alla scelta dello zero dello strumento e
      $\sigma_i$ è l'incertezza associata alla $i$-esima misura. Abbiamo preso come valore di $\sigma_\text{risoluzione}$ la metà
      della più piccola variazione apprezzabile, ovvero: $\sigma_\text{risoluzione} = \frac 1 {20} V_\text{fondoscala}$. Si noti che le
      incertezze sullo zero e sulla misura sono considerate \emph{dipendenti}, e quindi vengono sommate linearmente.

      \subsection{Incertezze per il multimetro}\label{subsec:incertezze-multimetro}
      Le incertezze sulle misure del multimetro sono ricavate dal manuale di istruzioni del fornitore. Per il nostro multimetro,
      in tutto l'intervallo di misura, valgono $\sigma_V = \pm (0.3\% + 2\text{digit})$ e $\sigma_I = \pm (0.4\% + 2\text{digit})$, rispettivamente per le misure di tensione e corrente.


\section{Valori numerici delle misure}\label{sec:valori-misure}
\subsection{Dati per $I_b = 100mA$}\label{subsec:base-100}
\begin{table}[H]
  \centering
  \begin{tabular}[t]{c|c|c||c|c|c}
    \toprule
    $V$ ($mV$) & $V_\text{fondoscala}$ ($V$) & $\sigma_V$ ($V$) & $I$ ($mA$) & $I_\text{fondoscala}$ ($mA$) & $\sigma_I$ ($mA$)%
    \csvreader[
      head to column names,
    ]{../data/100mA.csv}{}% use head of csv as column names
    {\\\hline\V&\fondoscalaV&\sigmaV&\I&\fondoscalaI&\sigmaI}\\%
    \bottomrule
  \end{tabular}
  \caption{
    Dati raccolti per $I_b = 100mA$. Le incertezze sono riportate con una cifra significativa o
    con due cifre significative, quando la prima cifra è $1$.
  }
  \label{tab:valori-100}
\end{table}

\subsection{Dati per $I_b = 150mA$}\label{subsec:base-150}
\begin{table}[H]
  \centering
  \begin{tabular}[t]{c|c|c||c|c|c}
    \toprule
    $V$ ($V$) & $V_\text{fondoscala}$ ($V$) & $\sigma_V$ ($V$) & $I$ ($mA$) & $I_\text{fondoscala}$ ($mA$) & $\sigma_I$ ($mA$)%
    \csvreader[
      head to column names,
    ]{../data/150mA.csv}{}% use head of csv as column names
    {\\\hline\V&\fondoscalaV&\sigmaV&\I&\fondoscalaI&\sigmaI}\\%
    \bottomrule
  \end{tabular}
  \caption{
    Dati raccolti per $I_b = 150mA$. Le incertezze sono riportate con una cifra significativa o
    con due cifre significative, quando la prima cifra è $1$.
  }
  \label{tab:valori-150}
\end{table}

\subsection{Dati per $I_b = 200mA$}\label{subsec:base-200}
\begin{table}[H]
  \centering
  \begin{tabular}[t]{c|c|c||c|c|c}
    \toprule
    $V$ ($V$) & $V_\text{fondoscala}$ ($V$) & $\sigma_V$ ($V$) & $I$ ($mA$) & $I_\text{fondoscala}$ ($mA$) & $\sigma_I$ ($mA$)%
    \csvreader[
      head to column names,
    ]{../data/200mA.csv}{}% use head of csv as column names
    {\\\hline\V&\fondoscalaV&\sigmaV&\I&\fondoscalaI&\sigmaI}\\%
    \bottomrule
  \end{tabular}
  \caption{
    Dati raccolti per $I_b = 200mA$. Le incertezze sono riportate con una cifra significativa o
    con due cifre significative, quando la prima cifra è $1$.
  }
  \label{tab:valori-200}
\end{table}


%\bibliographystyle{unsrt} % We choose the "plain" reference style
%\bibliography{../electronicsLabRefs.bib}

\end{document}
