\section{Introduzione}\label{sec:scopo}
In elettronica digitale, è spesso necessario poter disporre di una funzione
a onda quadra, per governare le componenti di un circuito.
Un modo molto semplice per realizzarla è partire
da un'onda sinusoidale e \emph{tosarne} le estremità.
La tosatura è realizzata tramite due diodi, collegati ad un potenziometro.
Il potenziometro permette di regolare l'ampiezza dell'onda tosata, impostando il
potenziale soglia a cui si verifica la tosatura (ovvero, il potenziale a cui i
diodi invertono la loro polarizzazione).
Chiamiamo \emph{tensione di polarizzazione} $V_{R}$ la tensione in uscita dai potenziometri
e $V_{\gamma}$ la \emph{tensione di soglia} dei diodi.
Ci aspettiamo che l'onda tosata abbia un'ampiezza $V$ comparabile con $V = V_R + V_\gamma$.
La forma dell'onda tosata è regolata da una resistenza variabile $R$ collegata in serie con il
generatore di funzione: diminuendo la resistenza, si osserva che i tagli dell'onda
diventano meno netti.
