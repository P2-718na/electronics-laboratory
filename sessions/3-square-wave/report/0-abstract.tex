\begin{abstract}\label{sec:abstract}
  Lo scopo di questa prova è stato quello di realizzare un'approssimazione di un onda quadra
  e di misurarne alcune caratteristiche.
  Abbiamo realizzato un circuito tosatore che prende in input
  un onda sinusoidale e usa due diodi per \emph{tosarne} i picchi.
  L'effetto di tosatura può essere regolato tramite due potenziometri.
  Abbiamo svolto l'esperimento sia usando diodi al silicio che al germanio.
  Qualitativamente abbiamo osservato il comportamento atteso.
  Abbiamo misurato le seguenti caratteristiche dell'onda risultante:
  semiampiezza tosata (Si: $2.55V \pm 0.09V$, Ge: $2.10V \pm 0.08V$),
  tempo di salita (Si: $52.0\mu s \pm 1.9\mu s$, Ge: $48.0\mu s \pm 1.8\mu s$),
  resistenza critica (Si: $35.90k\Omega \pm 0.16k\Omega$, Ge: $34.33 \pm 0.16k\Omega$).
\end{abstract}

