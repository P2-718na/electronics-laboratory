\section{Conclusioni}\label{sec:conclusioni}
I valori della semiampiezza tosata corrispondono a quanto atteso\footnote{considerando $V_\gamma$ per silicio e germanio
rispettivamente di $0.6V$ e $0.2V$.}, entro
le incertezze strumentali.
L'effetto di $V_\gamma$ si ripercuote anche
sul tempo di salita: questo aumenta proporzionalmente con la semiampiezza tosata.
Il valore ottenuto è compatibile con quanto atteso:
infatti, se consideriamo un onda sinusoidale di periodo $1ms$ e ampiezza non tosata
così come riportata in tabella \ref{tab:risultati}, il tempo di salita
tra il 10\% e il 90\% del rispettivo valore di ampiezza tosata vale
$57\mu s$ e $48 \mu s$ per silicio e germanio, rispettivamente.
Questi valori (ottenuti analiticamente), sono compatibili con i valori sperimentali ottenuti.

Qualitativamente, il circuito si comporta come atteso: variando i valori di
$V_{R1}$ e $V_{R2}$ separatamente, si possono apprezzare tagli diversi nella
nella regione superiore e inferiore dell'onda (vedi figura \ref{fig:dati-raccolti}).
Inoltre, variare $R$ comporta una variazione della curvatura del taglio.
I valori di resistenza critica sono compatibili tra loro, così come atteso.
