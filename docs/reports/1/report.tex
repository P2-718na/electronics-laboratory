\documentclass[11pt, a4paper, twoside]{article}

\usepackage{fontspec}
\usepackage{blindtext}
\usepackage{geometry}
\usepackage{setspace}
\usepackage{titlesec}
\usepackage{indentfirst}
\usepackage{graphicx}
\usepackage[italian]{babel}
\usepackage{catchfile} % used in \getenv command
\usepackage{multicol}
\usepackage{amsmath}
\usepackage{subcaption}
\usepackage[hang, flushmargin, multiple, bottom]{footmisc}
\usepackage{float}
\usepackage{array}
\usepackage{booktabs}
\usepackage{url}

\raggedbottom

\titlespacing*{\section}{0px}{3mm}{1mm}
\titlespacing*{\subsection}{0px}{3mm}{1mm}
\geometry{
  left=2cm,
  right=2cm,
  top=2cm,
  bottom=2cm
}
\setlength{\parindent}{10mm}
\graphicspath{ {./assets}, {../../assets} }

% Allow use of command \getenv{VARNAME}.
% Taken from: https://tex.stackexchange.com/questions/62010/can-i-access-system-environment-variables-from-latex-for-instance-home
\newcommand{\getenv}[2][]{
  \CatchFileEdef{\temp}{"|kpsewhich --var-value #2"}{\endlinechar=-1}%
  \if\relax\detokenize{#1}\relax\temp\else\let#1\temp\fi}

% Roman numerals
\newcommand{\rom}[1]{\uppercase\expandafter{\romannumeral #1\relax}}

\begin{document}

%\begin{center}
{
	{\Large
		{\textsc{Alma Mater Studiorum $\cdot$ Università di Bologna}}
	}
}
\rule[0.1cm]{18cm}{0.1mm}
\rule[0.5cm]{18cm}{0.6mm}
{\small
	{\bf SCUOLA DI SCIENZE\\
		Corso di Laurea in Fisica
	}
}
{\let\newpage\relax\maketitle}
\end{center}


\begin{center}
  \huge Misura della caratteristica I-V di due diodi a semiconduttore \\
  \large Giuseppe Sguera \getenv{MAT1}. Matteo Bonacini \getenv{MAT2}.\\ % see readme on how to use this
  \normalsize Turno 1, tavolo (numero non indicato).\\
  \today
\end{center}

\begin{abstract}\label{sec:abstract}
  In questa prova abbiamo misurato l'andamento I-V di due diodi a semiconduttore diversi (Silicio e Germanio), per poi svolgere un \emph{fit}
  sui dati raccolti per ricavare i parametri $I_0$ e $\eta V_T$. La prova è servita anche come addestramento per usare
  un oscilloscopio analogico. I risultati dei \emph{fit} sono ?? %todo compatibili/non.
\end{abstract}

\section{Introduzione}\label{sec:scopo}
  La caratteristica I-V di un diodo al silicio contiene tutte le informazioni necessarie per descriverne il comportamento.
  Per correnti piccole, l'andamento previsto dalla teoria è riassunto nnell'Equazione di Shockley
  \begin{equation}
    I(V) = I_0 \left(
      e^{
        \frac {V_d} {\eta V_t}
      } - 1
    \right)
    \label{eq:shockley}
  \end{equation}
  dove $V_d$ è la tensione applicata ai capi del diodo, $\eta$ è il \emph{fattore di idealità}, $V_T$ è un parametro dipendente dalla
  temperatura e $I_0$ è il valore numerico della corrente inversa. % todo add ref to libro da cui la scioli ha rubato i grafici
  % todo che cabbo è la corrente inversa porcoddio
  Aumentando la corrente oltre una certa soglia, l'andamento diventa sub-esponenziale e non rispetta più l'equazione \eqref{eq:shockley}.

  Lo scopo principale di questa prova è di misurare l'andamento I-V di due diodi a semiconduttore diversi, per poi svolgere un % todo add ref to materiale per diodi 
  \emph{fit} del loro andamento nel limite della validità dell'equazione \eqref{eq:shockley}. Dal fit verranno estratti i parametri $I_0$ e $\eta V_T$. 
  In secondo luogo, questa prova serve anche come addestramento per imparare ad usare un oscilloscopio analogico, in funzione delle successive
  prove di laboratorio.


\section{Apparato sperimentale}\label{sec:apparato-sperimentale}
% fixme svolgimento?
\subsection{Schema del circuito}\label{subsec:schema-circuito}

\subsection{Materiale usato}\label{subsec:materiale-usato}

\subsection{Strumenti usati}\label{subsec:strumenti-usati}


\section{Risultati}\label{sec:risultati}
% fixme conclusioni

\newpage
\section{Appendice A: valori numerici delle misure}\label{sec:valori-misure}


\newpage
\section{Appendice B: grafici}\label{sec:grafici}


\end{document}
